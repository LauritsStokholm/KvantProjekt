A method of approximating solutions of the one-dimensional Schrödinger equation is presented in this paper. The method closely resemles the usual WKB approximation, but differes only in the coice of basis whereas in this paper the solutions of an arbitrary Schrödinger equation are used.

This paper examines the hydrogen atom\ldots\ldots\ldots




This paper examines the behavior of the acousto optics effect, by using a  laser and an acousto-optic modulator(AOM). The behaviour of the light after modulation will greatly depend on the speed of sound within the AOM. It’s assumed this parameter is unknown. This parameter will be determined by two different approaches. By using the relations from the $\theta_{\text{sep}}$ And simple measure of diffraction distance, we found that the speed is . By using the relations between the rise and falltime of the switch and the speed of sound, $v_S$ was determined to be.

 It’s assumed the results are reaching the true value for the speed of sound, but that the measurements and calculations contains errors and uncertainties.


