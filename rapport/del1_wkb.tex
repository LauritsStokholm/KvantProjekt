
\section{WKB approksimation i brug ved Schrödingerligningen}
I dette afsnit vil vi betragte Schrödingerligningen for en partikel i én dimension. Introduceres den klassiske impuls, som er givet ved $p(x) = \sqrt{2m(E-V(x)}$, kan man omskrive ligningen.
%
\begin{align}
    & \frac{\hbar^2}{2m} \pdiff[2]{\psi(x)}{x} + V(x)\psi(x) = E\psi(x), \\
    & \Leftrightarrow \pdiff[2]{\psi(x)}{x} = \frac{2m}{\hbar^2}\psi(x)(E - V(x))  = - \frac{p(x)^2}{\hbar^2}\psi(x).
    \label{eq:schrodingerligning}
\end{align}
%
Det er kendt, at $\psi(x) \in \mathbb{C}$ er en kompleksfunktion som kan skrives som et produkt af en reel amplitude, $A \in \mathbb{R}$ og en kompleks eksponential med reel fase $\phi(x)\in \mathbb{R}$. Derfor anvendes nu ansatzen.
\begin{equation}
    \psi(x) = A(x) e^{i \phi(x)}.
    \label{eq:ansatz}
\end{equation}
Lad $E > V(x) \ \forall x$, hvilket svarer til et klassisk regime.
Nu kan $\pdiff[2]{\psi(x)}{x}$ bestemmes. For overskueligheden lader vi $A'(x) = \pdiff{A}{x}$, og så fremdeles af højere ordens led.
\begin{align}
    \pdiff[2]{\psi(x)}{x} = & \pdiff{}{x}\left( \left[ A'(x) + A(x)i\phi'(x) \right]e^{i\phi(x)} \right) \\
    = & \left[ A''(x) + A'(x)i\phi'(x) + A'(x)i\phi'(x) + A(x)i\phi''(x) - A(x)( \phi'(x))^2\right]e^{i\phi(x)} \\
    = & \left(\left[ A''(x) - A(x)(\phi'(x))^2\right] + i \left[2A'(x)\phi'(x) + A(x)\phi''(x) \right]\right)e^{i\phi(x)}\\
    = & -\frac{p^2}{\hbar^2}\psi(x) = -\frac{p^2}{\hbar^2} A(x)e^{i\phi(x)}.
    \label{eq:regime}
\end{align}
Hvor sidste udtryk kommer fra \cref{eq:schrodingerligning} sammensat med \cref{eq:ansatz}.
Denne andenordens differentialligning kan opdeles i sin real-- og imaginærdel. Desuden kan det udnyttes, at $\left( A(x)^2\phi'(x) \right)' = 2A'(x)\phi'(x) + A^2(x)\phi''(x)$, hvorfor

\begin{align}
    \overbrace{A''(x) - A(x)\left( \phi'(x) \right)^{2} = - \frac{p^{2}}{\hbar^{2}}A(x)}^{\text{Realdel}} \quad \text{og} \quad%
    \overbrace{\left( A(x)^{2}\phi'(x) \right)' = 0}^{\text{Imaginærdel}}.
    \label{eq:realogimag}
\end{align}
Heraf ses det fra den imaginære del, at det indre i parentesen er konstant, hvorfor
\begin{align}
    \left( A(x)^{2}\phi'(x) \right)' = 0  \quad & \Leftrightarrow  \quad A(x)^{2}\phi'(x) = C^{2} \in \mathbb{C} \\
    & \Leftrightarrow  \quad A(x) = \frac{C}{\sqrt{\phi'(x)}}.
    \label{eq:konst}
\end{align}
Desuden, ved at definere $k(x) = \frac{p(x)}{\hbar}$, kan realdelen omskrives. Indtil videre er der regnet eksakt, men denne del af differentialligningen er svær eller endda umulig at løse. Derfor antages $A$ som værende langsomt varierende, hvorfor krumningen er $A''(x)\ll 1$ og kan  negligeres.
Specielt gælder det at
\begin{equation}
  \frac{A''(x)}{A(x)} \ll \phi'(x)^{2}\wedge k(x)^2.
  \label{eq:TheApprox}
\end{equation}
Med antagelsen i \cref{eq:TheApprox} vil realdelen i \cref{eq:konst} reducere til
\begin{align}
    0 = \frac{A''(x)}{A(x)} = -k(x)^{2} + (\phi'(x))^{2}
    \quad \Leftrightarrow\quad  \left(\phi'(x)\right)^{2} = k(x)^{2} \quad \Leftrightarrow \quad  \phi'(x) = \pm \frac{p}{\hbar}.
    \label{eq:ho}
\end{align}
Hvis der integreres med hensyn til x i \cref{eq:ho} findes
%
\begin{equation}
    \phi(x) = \pm \frac{1}{\hbar} \int p(x) dx
    \label{eq:thisPhi}
\end{equation}
%
Substitueres $A(x)$ fra \cref{eq:konst} og udtrykket for $\phi{'\left (x \right )}$ fra \cref{eq:ho}  ind i ansatzen fra \cref{eq:ansatz} findes
%TODO \cong står for congruent (dansk kongruent). Måske vi bare skulle bruge \approx i stedet for.
\begin{equation}
    \psi(x) = A(x) e^{i \phi(x)} \cong
    \frac{C}{\sqrt{( \phi{'\left (x \right )})}} e^{i \phi(x)} \cong
    \frac{C}{\sqrt{\frac{p(x)}{\hbar}}} e^{\pm \frac{i}{\hbar} \int p(x) dx} \cong
    \frac{C'}{\sqrt{p(x)}} e^{\pm \frac{i}{\hbar} \int p(x) dx}
    \label{eq:final}
\end{equation}
%
Hvor $C'$ er en ny konstant, der indeholder $\hbar$ og evt. andre konstanter fra integralet. Dette er det endelige udtryk for WKB-approksimationen. Løsningen ville da være en linearkombination af den positive og negative eksponent. Det bemærkes ved Borns statistiske fortolkning, at sansynligheden for at partiklen måles ved et givent $x$ er omvendt proportionel med dens klassiske impuls (se \cref{eq:final}).
\\
Når $\psi(x)$ nærmer sig de klassiske vendepunkter, vil betingelsen fra \cref{eq:TheApprox} ikke gælde, da $E \approx V(x)$. Her er approksimationen ikke blot upræcis, men faktisk ugyldig. $E \rightarrow V(x)$ betyder at $p(x) \rightarrow 0$ og det ses fra \cref{eq:final}  at $\psi(x) \rightarrow \infty$. Løsningen her kan ikke normeres, hvorfor det ikke er en fysisk løsning. For at løse problemet ved de klassiske vendepunkter, anvendes såkaldte "lappe-ligninger" ved grænserne, men det er ikke formålet med denne rapport at vise disse.
