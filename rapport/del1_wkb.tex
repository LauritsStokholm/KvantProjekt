\section{WKB approksimation i brug ved Schrödingerligningen}
I dette afsnit vil vi betragte Schrödingerligningen for en partikel i én dimension. Introduceres den klassiske impuls, som er givet ved $p(x) = \sqrt{2m(E-V(x)}$, kan man omskrive ligningen.
%
\begin{align}
    & \frac{\hbar^2}{2m} \pdiff[2]{\psi(x)}{x} + V(x)\psi(x) = E\psi(x) \\
    & \Leftrightarrow \pdiff[2]{\psi(x)}{x} = \frac{2m}{\hbar^2}\psi(x)(E - V(x))  = - \frac{p(x)^2}{\hbar^2}\psi(x).
    \label{eq:schrodingerligning}
\end{align}
%
Det er kendt, at $\psi(x) \in \mathbb{C}$ er en kompleksfunktion som kan skrives som et produkt af en reel amplitude, $A \in \mathbb{R}$ og en kompleks eksponential med reel fase $\phi(x)\in \mathbb{R}$. Derfor anvendes nu ansatzen.
\begin{equation}
    \psi(x) = A(x) e^{i \phi(x)}
    \label{eq:ansatz}
\end{equation}
Lad $E > V(x) \ \forall x$, hvilket svarer til et klassisk regime.
Nu kan $\pdiff[2]{\psi(x)}{x}$ bestemmes. For overskueligheden lader vi $A'(x) = \pdiff{A}{x}$, og så fremdeles af højere ordens led.
\begin{align}
    \pdiff[2]{\psi(x)}{x} = & \pdiff{}{x}\left( \left[ A'(x) + A(x)i\phi'(x) \right]e^{i\phi(x)} \right), \\
    = & \left[ A''(x) + A'(x)i\phi'(x) + A'(x)i\phi'(x) + A(x)i\phi''(x) - A(x)( \phi'(x))^2\right]e^{i\phi(x)}, \\
    = & \left(\left[ A''(x) - A(x)(\phi'(x))^2\right] + i \left[2A'(x)\phi'(x) + A(x)\phi''(x) \right]\right)e^{i\phi(x)},\\
    = & -\frac{p^2}{\hbar^2}\psi(x) = -\frac{p^2}{\hbar^2} A(x)e^{i\phi(x)}.
    \label{eq:regime}
\end{align}
Hvor sidste udtryk kommer fra \cref{eq:schrodingerligning} sammensat med \cref{eq:ansatz}.
Denne andenordens differentialligning kan opdeles i sin real-- og imaginærdel. Desuden kan det udnyttes, at $\left( A(x)^2\phi'(x) \right)' = 2A'(x)\phi'(x) + A^2(x)\phi''(x)$, hvorfor

\begin{align}
    \overbrace{A''(x) - A(x)\left( \phi'(x) \right)^{2} = - \frac{p^{2}}{\hbar^{2}}A(x)}^{\text{Realdel}} \quad \text{og} \quad%
    \overbrace{\left( A(x)^{2}\phi'(x) \right)' = 0}^{\text{Imaginærdel}}.
    \label{eq:realogimag}
\end{align}
Heraf ses det fra den imaginære del, at det indre i parentesen er konstant, hvorfor
\begin{align}
    \left( A(x)^{2}\phi'(x) \right)' = 0  \quad & \Leftrightarrow  \quad A(x)^{2}\phi'(x) = C^{2} \in \mathbb{C} \\
    & \Leftrightarrow  \quad A(x) = \frac{C}{\sqrt{\phi'(x)}}
    \label{eq:konst}
\end{align}
Desuden, ved at definere $k(x) = \frac{p(x)}{\hbar}$, kan realdelen omskrives. Indtil videre er der regnet eksakt, men denne del af differentialligning er svær at løse. Derfor antages $A$ som værende langsomt varierende, hvorfor krumningen er $A''(x)\ll 1$ og kan  negligeres. Specielt er $\frac{A''(x)}{A(x)} \ll \phi'(x)^{2}\wedge \frac{p^{2}}{\hbar^{2}}$.
\begin{align}
    A''(x) - A(x)(\phi'(x))^{2} & =  -k(x)^{2}A(x),\\
    \Leftrightarrow \frac{A''(x)}{A(x)} & = -k(x)^{2} + (\phi'(x))^{2} = 0, \\
    \Leftrightarrow \left(\phi'(x)\right)^{2} & = k(x)^{2} \Leftrightarrow  \phi'(x) = \pm \frac{p}{\hbar}.
    \label{eq:ho}
\end{align}
hvorfor $\phi = \pm \frac{1}{\hbar}\int p(x) dx$. Disse resultater fra \cref{eq:konst} og \cref{eq:ho} indsættes \cref{eq:ansatz}.
\begin{align}
    \psi(x) = \frac{C}{\sqrt{\phi'(x)}}e^{\pm i\frac{1}{\hbar}\int p(x) dx}.
    \label{eqpsi}
\end{align}
Den generelle WKB-løsning er en linearkombination af disse $\psi(x)$. Det bemærkes ved Borns statistiske fortolkning, at sansynligheden for at partiklen måles ved et givent $x$ er omvendt proportionel med dens klassiske impuls; $|\psi(x)|^2 \approx \frac{|C|^2}{\phi'(x)}$.


%Dette må også være lig med højre side af \cref{eq:scrodingerLigning3}. Substitueres udtrykket for $\psi$ fra \cref{eq:ansatz} kan \cref{eq:diff2gange} reduceres til
%
%\begin{align}
%    \left(- A{\left (x \right )} \left(\frac{d}{d x} \phi{\left (x \right )}\right)^{2} + i A{\left (x \right )} \frac{d^{2}}{d x^{2}}  \phi{\left (x \right )} + 2 i \frac{d}{d x} A{\left (x \right )} \frac{d}{d x} \phi{\left (x \right )} + \frac{d^{2}}{d x^{2}}  A{\left (x \right )}\right) e^{i \phi{\left (x \right )}} = - \frac{p^2}{\hbar^2} \psi = - \frac{p(x)^2}{\hbar^2} A(x) e^{i \phi(x)}
%    \label{eq:udskrevet}
%\end{align}
%
%\begin{equation}
%     \Rightarrow - A{\left(x \right)} \left(\frac{d}{d x} \phi{\left(x \right)}\right)^{2} + i A{\left(x \right)} \frac{d^{2}}{d x^{2}}  \phi{\left(x \right)} + 2 i \frac{d}{d x} A{\left(x \right)} \frac{d}{d x} \phi{\left (x \right )} + \frac{d^{2}}{d x^{2}}  A{\left (x \right )}
%     = - \frac{p(x)^2}{\hbar^2} A(x)
%     \label{eq:ReOgIm}
%\end{equation}
%
%Sorteres venstre side af \cref{eq:ReOgIm} real delen og imaginær delen fås henholdsvis
%
%\begin{equation}
%    A{\left (x \right )} \left(\frac{d}{d x} \phi{\left (x \right )}\right)^{2}  + \frac{d^{2}}{d x^{2}}  A{\left (x \right )}
%    = - \frac{p(x)^2}{\hbar^2} A(x)
%    \label{eq:Re}
%\end{equation}
%
%\begin{equation}
%    i A{\left (x \right )} \frac{d^{2}}{d x^{2}}  \phi{\left (x \right )} + 2 i \frac{d}{d x} A{\left (x \right )} \frac{d}{d x} \phi{\left (x \right )}
%    = - \frac{p(x)^2}{\hbar^2} A(x)
%    \label{eq:Im}
%\end{equation}
%
%Af disse kan\cref{eq:Im}  løses, løsningen er: % er det rigtigt? Hvorfor er der ingen løsning i Realdelen
%
%\begin{equation}
%    A(x) = \frac{C}{\sqrt{(\frac{d}{d x} \phi{\left (x \right )})}}.
%    \label{eq:ImLoes}
%\end{equation}
%
%Hvor C er en konstant. Der eksisterer ingen løsninger til \cref{eq:Re}, og derfor foretages en approximation, ved blot at droppe $\frac{d^{2}}{d x^{2}}  A{\left (x \right )}$ leddet i samme ligning. Dette er en god approixmation, hvis $A(x)$ varier langsomt. Da kan \cref{eq:Re} omskrives til
%
%\begin{equation}
%    \left(\frac{d}{d x} \phi{\left (x \right )}\right)^{2}
%    = - \frac{p(x)^2}{\hbar^2}
%    \label{eq:ReApprox}
%\end{equation}
%
%\begin{equation}
%    \Rightarrow \frac{d}{d x} \phi{\left (x \right )}
%    = \pm \frac{p(x)}{\hbar}
%    \label{eq:ReApproxFinal}
%\end{equation}
%
%Hvis der integreres med hensyn til x findes
%
%\begin{equation}
%    \phi(x) = \pm \frac{1}{\hbar} \int p(x) dx
%    \label{wq:thisPhi}
%\end{equation}
%
%Substitueres $A(x)$ fra \cref{eq:ImLoes} og udtrykket for $\frac{d}{d x} \phi{\left (x \right )}$ fra \cref{eq:ReApproxFinal}  ind i ansatzen fra \cref{eq:ansatz} findes
%
%\begin{equation}
%    \psi(x) = A(x) e^{i \phi(x)} \cong
%    \frac{C}{\sqrt{(\frac{d}{d x} \phi{\left (x \right )})}} e^{\phi(x)} \cong
%    \frac{C}{\sqrt{\frac{p(x)}{\hbar}}} e^{\pm \frac{1}{\hbar} \int p(x) dx} \cong
%    \frac{C'}{\sqrt{p(x)}} e^{\pm \frac{1}{\hbar} \int p(x) dx}
%    \label{eq:final}
%\end{equation}
%
%Hvor $C'$ er en ny konstant, der indeholder $\hbar$ og evt. andre konstanter fra integralet. Dette er det endelige udtryk for aproximationen. Løsningen ville da være en linear kombination af den positive og negative del.
