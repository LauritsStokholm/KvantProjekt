\section{WKB approksimation i brug ved Schrödingerligningen}
I dette afsnit vil vi betragte Schrödingerligningen for en partikel i én dimension. Derved er Schrödinger ligningen
%
\begin{equation}
    \frac{\hbar^2}{2m}\frac{\partial^2 \psi}{\partial x^2} + V(x) \psi = E \psi
    \label{eq:scrodingerLigning1}
\end{equation}
%
isoleres $\frac{\partial^2 \psi}{\partial x^2}$ i \cref{eq:scrodingerLigning1}, opnås
%
\begin{equation}
    \frac{\partial^2 \psi}{\partial x^2} = 2m\psi (E  - V(x)) \frac{1}{\hbar^2}.
    \label{eq:scrodingerLigning2}
\end{equation}
%
Hvis der defienres $p(x) \equiv \sqrt{2m(E-V(x))}$ fra det klassiske regime, Kan \cref{eq:scrodingerLigning2} omskrives til
%
\begin{equation}
    \frac{\partial^2 \psi}{\partial x^2} = - \frac{p(x)^2}{\hbar^2} \psi.
    \label{eq:scrodingerLigning3}
\end{equation}
%
$\psi(x) \in \mathbb{C}$ er en kompleksfunktion som kan skrives som et produkt af en reel amplitude, $A \in \mathbb{R}$ og en kompleks eksponential med reel fase $\phi(x)\in \mathbb{R}$. Derfor anvendes nu ansatzen.
%
\begin{equation}
    \psi(x) = A(x) e^{i \phi(x)}.
    \label{eq:ansatz}
\end{equation}
%
Hvis det gælder at $ E > V(x) \forall x$ Er $A(x)$ en reel amplitude og $\phi(x)$ er en reel fase. Dette kan altid gøres, der man ved hjælp af ledet  $e^{i \phi(x)} $. Dette led danner en vektor $ \in \mathbb{C}$ med normen 1. Herefter kan $A(x)$  skalere vektoren, til at ramme alle punkter. Anvendes venstre side af \cref{eq:scrodingerLigning3} på \cref{eq:ansatz} fås
%
\begin{equation}
    \frac{\partial^2 \psi}{\partial x^2} = \frac{\partial \psi}{\partial x} \left( \left(i A{\left (x \right )} \frac{d}{d x} \phi{\left (x \right )} + \frac{d}{d x} A{\left (x \right )}\right) e^{i \phi{\left (x \right)}} \right)
    \label{eq:diff1gange}
\end{equation}
%
\begin{align}
    = \left(- A{\left (x \right )} \left(\frac{d}{d x} \phi{\left (x \right )}\right)^{2} + i A{\left (x \right )} \frac{d^{2}}{d x^{2}}  \phi{\left (x \right )} + 2 i \frac{d}{d x} A{\left (x \right )} \frac{d}{d x} \phi{\left (x \right )} + \frac{d^{2}}{d x^{2}}  A{\left (x \right )}\right) e^{i \phi{\left (x \right )}}
    \label{eq:diff2gange}
\end{align}
%
Dette må også være lig med højre side af \cref{eq:scrodingerLigning3}. Substitueres udtrykket for $\psi$ fra \cref{eq:ansatz} kan \cref{eq:diff2gange} reduceres til
%
\begin{align}
    \left(- A{\left (x \right )} \left(\frac{d}{d x} \phi{\left (x \right )}\right)^{2} + i A{\left (x \right )} \frac{d^{2}}{d x^{2}}  \phi{\left (x \right )} + 2 i \frac{d}{d x} A{\left (x \right )} \frac{d}{d x} \phi{\left (x \right )} + \frac{d^{2}}{d x^{2}}  A{\left (x \right )}\right) e^{i \phi{\left (x \right )}} = - \frac{p(x)^2}{\hbar^2} A(x) e^{i \phi(x)}
    \label{eq:udskrevet}
\end{align}
%
\begin{equation}
     \Rightarrow
     \overbrace{- A{\left(x \right)} \left(\frac{d}{d x} \phi{\left(x \right)}\right)^{2}
     + \frac{d^{2}}{d x^{2}}  A{\left (x \right )}}^{\text{Realdel}}
     \overbrace{+ i A{\left(x \right)} \frac{d^{2}}{d x^{2}}  \phi{\left(x \right)}
     + 2 i \frac{d}{d x} A{\left(x \right)} \frac{d}{d x} \phi{\left (x \right )}}^{\text{Imaginærdel}}
     = - \frac{p(x)^2}{\hbar^2} A(x)
     \label{eq:ReOgIm}
\end{equation}
%
Som vist i ligning \cref{eq:ReOgIm} består $- \frac{p(x)^2}{\hbar^2} A(x)$ af en realdel og en imaginærdel. Af disse, er det kun realden der kan løses, som er
%
\begin{equation}
    A(x) = \frac{C}{\sqrt{(\frac{d}{d x} \phi{\left (x \right )})}}.
    \label{eq:ImLoes}
\end{equation}
%
Hvor C er en konstant. Der eksisterer ingen løsninger til realdelen, og derfor foretages en approximation, ved blot at droppe $\frac{d^{2}}{d x^{2}}  A{\left (x \right )}$. Dette er en god approixmation, hvis $A(x)$ varier langsomt. Eller mere præcis skal det gælde:
\begin{equation}
    \frac{A''(x)}{A(x)} \ll \phi'(x)^{2}\wedge \frac{p^{2}}{\hbar^{2}}
    \label{eq:Theapprox}
\end{equation}
%
Hvis \cref{eq:Theapprox} gælder, kan realdelen i \cref{eq:ReOgIm} omskrives til
%
\begin{equation}
    \left(\frac{d}{d x} \phi{\left (x \right )}\right)^{2}
    = - \frac{p(x)^2}{\hbar^2}
    \label{eq:ReApprox}
\end{equation}
%
\begin{equation}
    \Rightarrow \frac{d}{d x} \phi{\left (x \right )}
    = \pm \frac{p(x)}{\hbar}
    \label{eq:ReApproxFinal}
\end{equation}
%
Hvis der integreres med hensyn til x i \cref{eq:ReApproxFinal} findes
%
\begin{equation}
    \phi(x) = \pm \frac{1}{\hbar} \int p(x) dx
    \label{eq:thisPhi}
\end{equation}
%
Substitueres $A(x)$ fra \cref{eq:ImLoes} og udtrykket for $\frac{d}{d x} \phi{\left (x \right )}$ fra \cref{eq:ReApproxFinal}  ind i ansatzen fra \cref{eq:ansatz} findes
%
\begin{equation}
    \psi(x) = A(x) e^{i \phi(x)} \cong
    \frac{C}{\sqrt{(\frac{d}{d x} \phi{\left (x \right )})}} e^{\phi(x)} \cong
    \frac{C}{\sqrt{\frac{p(x)}{\hbar}}} e^{\pm \frac{1}{\hbar} \int p(x) dx} \cong
    \frac{C'}{\sqrt{p(x)}} e^{\pm \frac{1}{\hbar} \int p(x) dx}
    \label{eq:final}
\end{equation}
%
Hvor $C'$ er en ny konstant, der indeholder $\hbar$ og evt. andre konstanter fra integralet. Dette er det endelige udtryk for aproximationen. Løsningen ville da være en linear kombination af den positive og negative del. Iflg. Borns statistiske fortolkning tages normkvadratet af \label{eq:final}, finder man at $|\psi(x)|^2 \approx \frac{|C|^2}{\phi'(x)}$. Med dette kan man bl.a. se, at det er usandsynligt at finde partiklen, der hvor hastigheden er stor.
\\
Bemærk, at når betingelsen fra \cref{eq:ReApprox} ikke gælder, altså når et pnukt nærmes de klassiske vendepunkter $E \approx V(X)$, er approximationen ikke blot upræcis, men derimod ugyldig. $E \approx V(X)$ betyder at $p(x) \rightarrow 0$ og det ses fra \cref{eq:final}  at $\psi(x) \rightarrow \infty$. Dette kan ikke normaliseres. For at løse problemet ,anvendes såkaldte "lappe-områder". Men det er ikke formålet med denne rapport.
