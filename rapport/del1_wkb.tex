\section{Solution to the stationary Schrödinger Equation}

Antages, at der betragtes en partikel i en dimension, $x$,  er Schrödinger ligningen

\begin{equation}
    \frac{\hbar^2}{2m}\frac{\partial^2 \psi}{\partial x^2} + V(x) \psi = E \psi
    \label{eq:scrodingerLigning1}
\end{equation}

isoleres $\frac{\partial^2 \psi}{\partial x^2}$ i \cref{eq:scrodingerLigning1}, opnås

\begin{equation}
    \frac{\partial^2 \psi}{\partial x^2} = 2m\psi (E  - V(x)) \frac{1}{\hbar^2}
    \label{eq:scrodingerLigning2}
\end{equation}

defineres $p(x)$ klassisk

\begin{equation}
p(x) \equiv \sqrt{2m(E-V(x))}
\end{equation}
Kan \cref{eq:scrodingerLigning2} omskrives til

\begin{equation}
    \frac{\partial^2 \psi}{\partial x^2} = - \frac{p(x)^2}{\hbar^2} \psi.
    \label{eq:scrodingerLigning3}
\end{equation}

Anvendes ansatzen

\begin{equation}
    \psi(x) = A(x) e^{i \phi(x)}
    \label{eq:ansatz}
\end{equation}

hvor $\psi (x) \in \mathbb{C}$. Antages

\begin{equation}
 E > V(x) \forall x
\end{equation}

Er $A(x)$ en reel amplitude og $\phi(x)$ er en reel fase. Dette kan altid gøres, der man ved hjælp af ledet  $e^{i \phi(x)} $. Dette led danner en vektor $ \in \mathbb{C}$ med normen 1. Herefter kan $A(x)$  skalere vektoren, til at ramme alle punkter. Anvendes venstre side af \cref{eq:scrodingerLigning3} på \cref{eq:ansatz} fås

\begin{equation}
    \frac{\partial \psi}{\partial x} = \left(i A{\left (x \right )} \frac{d}{d x} \phi{\left (x \right )} + \frac{d}{d x} A{\left (x \right )}\right) e^{i \phi{\left (x \right )}}
    \label{eq:diff1gange}
\end{equation}

\begin{align}
    \frac{\partial^2 \psi}{\partial x^2} = \left(- A{\left (x \right )} \left(\frac{d}{d x} \phi{\left (x \right )}\right)^{2} + i A{\left (x \right )} \frac{d^{2}}{d x^{2}}  \phi{\left (x \right )} + 2 i \frac{d}{d x} A{\left (x \right )} \frac{d}{d x} \phi{\left (x \right )} + \frac{d^{2}}{d x^{2}}  A{\left (x \right )}\right) e^{i \phi{\left (x \right )}}
    \label{eq:diff2gange}
\end{align}

Dette må også være lig med højre side af \cref{eq:scrodingerLigning3}. Substitueres udtrykket for $\psi$ fra \cref{eq:ansatz} kan \cref{eq:diff2gange} reduceres til

\begin{align}
    \left(- A{\left (x \right )} \left(\frac{d}{d x} \phi{\left (x \right )}\right)^{2} + i A{\left (x \right )} \frac{d^{2}}{d x^{2}}  \phi{\left (x \right )} + 2 i \frac{d}{d x} A{\left (x \right )} \frac{d}{d x} \phi{\left (x \right )} + \frac{d^{2}}{d x^{2}}  A{\left (x \right )}\right) e^{i \phi{\left (x \right )}} = - \frac{p^2}{\hbar^2} \psi = - \frac{p(x)^2}{\hbar^2} A(x) e^{i \phi(x)}
    \label{eq:udskrevet}
\end{align}

\begin{equation}
     \Rightarrow - A{\left (x \right )} \left(\frac{d}{d x} \phi{\left (x \right )}\right)^{2} + i A{\left (x \right )} \frac{d^{2}}{d x^{2}}  \phi{\left (x \right )} + 2 i \frac{d}{d x} A{\left (x \right )} \frac{d}{d x} \phi{\left (x \right )} + \frac{d^{2}}{d x^{2}}  A{\left (x \right )}}
     = - \frac{p(x)^2}{\hbar^2} A(x)
     \label{eq:ReOgIm}
\end{equation}

Sorteres venstre side af \cref{eq:ReOgIm} real delen og imaginær delen fås henholdsvis

\begin{equation}
    A{\left (x \right )} \left(\frac{d}{d x} \phi{\left (x \right )}\right)^{2}  + \frac{d^{2}}{d x^{2}}  A{\left (x \right )}}
    = - \frac{p(x)^2}{\hbar^2} A(x)
    \label{eq:Re}
\end{equation}

\begin{equation}
    i A{\left (x \right )} \frac{d^{2}}{d x^{2}}  \phi{\left (x \right )} + 2 i \frac{d}{d x} A{\left (x \right )} \frac{d}{d x} \phi{\left (x \right )}
    = - \frac{p(x)^2}{\hbar^2} A(x)
    \label{eq:Im}
\end{equation}

Af disse kan\cref{eq:Im}  løses, løsningen er: %TODO er det rigtigt? Hvorfor er der ingen løsning i Realdelen

\begin{equation}
    A(x) = \frac{C}{\sqrt{(\frac{d}{d x} \phi{\left (x \right )})}}.
    \label{eq:ImLoes}
\end{equation}

Der eksisterer ingen løsninger til \cref{eq:Re}, og derfor foretages en approximation, ved blot at droppe $\frac{d^{2}}{d x^{2}}  A{\left (x \right )}}$ leddet i samme ligning. Dette er en god approixmation, hvis $A(x)$ varier langsomt. Da kan \cref{eq:Re} omskrives til

\begin{equation}
    \left(\frac{d}{d x} \phi{\left (x \right )}\right)^{2}
    = - \frac{p(x)^2}{\hbar^2}
    \label{eq:ReApprox}
\end{equation}

\begin{equation}
    \Rightarrow \frac{d}{d x} \phi{\left (x \right )}
    = \pm \frac{p(x)}{\hbar}
    \label{eq:ReApproxFinal}
\end{equation}

%TODO TILFØJ ARGUMENT HVAD SKER DER HER

\begin{equation}
    \phi(x) = \pm \frac{1}{\hbar} \int p(x) dx
\end{equation}

%TODO TILFØJ ARGUMENT

\begin{equation}
    \psi(x) \cong \frac{C}{\sqrt{p(x)}} e^{\phi(x)} \cong \frac{C}{\sqrt{p(x)}} e^{\pm \frac{1}{\hbar} \int p(x) dx}
    \label{eq:final}
\end{equation}
