\section{Indledning}

% WKB metoden er en generel teori anvendt til at approksimere løsninger til differentialligninger, hvis højeste ordensled multipliceres af en \emph{lille} parameter $\varepsilon$. Dette kan specielt bruges til løsning af den én-dimensionale, tidsuafhængig Schrödingerligning. Hvor den almindelige WKB metode bruger eksponentialfunktionen som basis samt Langer's approksimationer til Bessel funktioner, vil vi i denne rapport vise, at man kan konstruere en generel WKB-approksimation hvis base er bestående af løsningerne til en arbitrær Schrödingerligning.\footnote{\url{https://journals-aps-org.ez.statsbiblioteket.dk:12048/pr/pdf/10.1103/PhysRev.91.174}}
% Vi vil i det følgende bestemme denne type af WKB-approksimation til Schrödingerligningen, og dernæst anvende den på brintatomet. Til sidst\ldots

Som bekendt fra det uendelige brøndpotentiale, bliver udregningerne nemmere, når man har et område med et konstant potentiale $V(x)$ at gøre. Virkeligheden er dog ikke altid så skånsom, og der kan opstå situationer der ikke er mulige at løse. Til gengæld kan man være heldig, at stå i en situation hvor $V(x)$ varierer langsomt i forhold til bølgelængden af $\psi$. I en sådan situation kan man vælge, blot at antage potentialet er konstant, og derved opnå nemmere udregninger. Dette er netop ideen bag WKB approksimationen.
Vi vil i det følgende udlede denne type af WKB-approksimation til Schrödingerligningen. Herefter vil vi anvende den til at udlede energi for et potentialbrønd med vertikale vægge, og afslutningsvist anvende WKB approksimationen på brintatomet, til at udlede de tilladte energier samt den asymptotiske form af bølgefunktionerne for brint.
