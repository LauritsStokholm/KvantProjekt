\section{Indledning}

WKB metoden er en generel teori anvendt til at approksimere løsninger til differentialligninger, hvis højeste ordensled multipliceres af en \emph{lille} parameter $\varepsilon$. Dette kan specielt bruges til løsning af den én-dimensionale, tidsuafhængig Schrödingerligning. Hvor den almindelige WKB metode bruger eksponentialfunktionen som basis samt Langer's approksimationer til Bessel funktioner, vil vi i denne artikel vise, at man kan konstruere en generel WKB-approksimation hvis base er bestående af løsningerne til en arbitrær Schrödingerligning.\footnote{\url{https://journals-aps-org.ez.statsbiblioteket.dk:12048/pr/pdf/10.1103/PhysRev.91.174}}
Vi vil i det følgende bestemme denne type af WKB-approksimation til Schrödingerligningen, og dernæst anvende den på brintatomet. Til sidst\ldots

% WKB metoden er en generel teori anvendt til at approksimere løsninger til differentialligninger, hvis højeste ordensled multipliceres af en \emph{lille} parameter $\varepsilon$. Dette kan specielt bruges til løsning af den én-dimensionale, tidsuafhængig Schrödingerligning og andre problemer som tunnerling, hvor der ikke nødvendigvis eksisterer en eksakt løsning. I denne rapport, vil vi undersøge udledningen af WKB-approximationen, og anvende den til at udlede de tilladet energi i brintatomet.
