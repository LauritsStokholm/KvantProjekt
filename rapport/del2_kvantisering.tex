\section{Potentialbrønd med vertikale vægge}
Lad os betragte det klassiske system, $E > V(x) \ \forall x \in[a,b]$, hvor intervallet $[a,b]$ er det begrænsede interval med $a$ og $b$ som klassiske vendepunkter. Heri eksisterer bølgefunktionen $\psi(x)$, ækvivalent til en brønd med vertikale vægge. Lad $|\psi(a)|^{2} = 0 = |\psi(b)|^{2}$. Fra \cref{eq:final} vil man kunne skrive $\psi(x)$ som en sum af hhv. negativ og positiv eksponent:
\begin{equation}
    \psi(x) \cong \frac{1}{\sqrt{p(x)}}\left[C_{+}e^{i\phi(x)}+C_{-}e^{-i\phi(x)}\right].
  \label{eq:kvantiseringStart}
\end{equation}
Ved at definere to nye konstanter, $C_1 \equiv i(C_{+}-C_{-})$ og $C_2 \equiv C_{+}+C_{-}$, kan dette ved brug af Eulers identitet skrives om til
\begin{equation}
  \psi(x) = \frac{1}{\sqrt{p(x)}}
  \Bigl[    C_1\sin{\phi(x)} + C_2\cos{\phi(x)}   \Bigr].
  \label{eq:kvantiseringSinCos}
\end{equation}
Her gælder det, at $\phi(x) = \frac{1}{\hbar}\int p(x) dx$. Integreres over intervallet $[a, x]$, vil
\begin{equation}
  \phi(x) = \frac{1}{\hbar}\int_{a}^{x} p(x')dx'.
\end{equation}
Fra startbetingelsen kræves at $\psi(a) = 0 = \psi(b)$. Da\footnote{Dette kan nemt ses, ved at integrere fra og til $a$} $\phi(a) = 0$, må $C_2$ være 0, da $\cos(0)\neq 0$. Dette medfører at,
\begin{equation}
    \psi(x) = \frac{C_1}{\sqrt{p(x)}}\sin{\phi(x)}.
\end{equation}
Nu skal det stadig gælde at $\psi(b) = 0$, og da\footnote{Da dette ellers ville være en ikke normerbar løsning, og en ret kedelig en af slagsen.} $C_1\neq 0$, må det altså betyde at $\sin(\phi(b)) = 0$, hvilket medfører at $\phi(b) = n'\pi$, hvor $n' \in \mathbb{Z}$. Da fortegnet kan absorberes i konstanten, så må $n' \in \mathbb{N} \setminus \{0\}$ bliver resultatet
\begin{equation}
    \phi(b) = n'\pi = \frac{1}{\hbar}\int_{a}^{b} p(x) dx \quad \Leftrightarrow \quad \int_{a}^{b} p(x) dx = n'\pi\hbar.
  \label{eq:kvantiDone}
\end{equation}
%TODO fjern evt dette energitjek ned i appendix hvis vi har for meget.
Vi kan tjekke validiteten af \cref{eq:kvantiDone}, ved at anvende den på den uendelige brønd. Vi har altså
\begin{align}
    V(x) =
    \begin{cases}
    0, & x\in]a, b[ \\
    \infty, & \text{ellers}
    \end{cases}
\end{align}
Kvantiseringsbetingelsen fra \cref{eq:kvantiDone} giver da, at
\begin{align}
    \int_{a}^{b} p(x) dx = & n\pi \hbar, \quad \text{hvor} \quad p(x) = \sqrt{2m(E-V_{\text{eff}})}\\
    \Leftrightarrow \sqrt{2mE}\int_{a}^{b} dx = & \sqrt{2mE}(b-a) = n\pi \hbar\\
    \Leftrightarrow E = & \frac{n^{2}\pi^{2}\hbar^{2}}{2m(b-a)^{2}},
    %\label{ja}
\end{align}
hvilket stemmer overens med \cite[s. 30]{griffiths}. Dette kan forventes, at da potentialet er $0$, vil approksimationen $A''(x) \approx 0$ fra \cref{eq:TheApprox} være eksakt, og derfor er dette resultat også eksakt.
