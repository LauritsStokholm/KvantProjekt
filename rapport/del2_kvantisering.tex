\section{Potentialebrønd med vertikale vægge}
Lad os betragte det klassiske system, $E > V(x) \ \forall x \in[a,b]$, hvor intervallet $[a,b]$ er det begrænsede interval med $a$ og $b$ som klassiske vendepunkter. Heri eksisterer bølgefunktionen $\psi(x)$, ækvivalent til en brønd med vertikale vægge. Lad $|\psi(a)|^{2} = 0 = |\psi(b)|^{2}$. Fra \cref{eqpsi} vil man kunne skrive $\psi(x)$ som en sum af hhv. negativ og positiv eksponent.

\begin{equation}
    \psi(x) \cong \frac{1}{\sqrt{p(x)}}\left[C_{+}e^{i\phi(x)}+C_{-}e^{-i\phi(x)}\right].
  \label{eq:kvantiseringStart}
\end{equation}
Ved at definere to nye konstanter, $C_1 \equiv i(C_{+}-C_{-})$ og $C_2 \equiv C_{+}+C_{-}$, kan dette ved brug af eulers identitet skrives om til
\begin{equation}
  \psi(x) = \frac{1}{\sqrt{p(x)}}
  \Bigl[    C_1\sin{\phi(x)} + C_2\cos{\phi(x)}   \Bigr].
  \label{eq:kvantiseringSinCos}
\end{equation}
Her gælder det, at $\phi(x) = \frac{1}{\hbar}\int p(x) dx$. Integreres over intervallet $[a, x]$, vil
\begin{equation}
  \phi(x) = \frac{1}{\hbar}\int_{a}^{x} p(x')dx'.
\end{equation}
Fra startbetingelsen kræves at $\psi(a) = \psi(b) = 0$. Da\footnote{Dette kan nemt ses, ved at integrere fra og til $a$} $\phi(a) = 0$, må $C_2$ være 0, da cosinus leddet ikke går i 0. Dette medfører at,
\begin{equation}
    \psi(x) = \frac{1}{\sqrt{p(x)}}C_3\sin{\phi(x)}.
\end{equation}
Nu skal det stadig gælde at $\psi(b) = 0$, og da altså må dette betyde at $sin(\phi(b)) = 0$ hvilket medfører at $\phi(b) = n'\pi$, hvor $n' = 0,1,2...$.
Dette giver os altså at

\begin{equation}
  \int_{a}^{b} p(x) dx = n'\pi\hbar
  \label{eq:kvantiDone}
\end{equation}
