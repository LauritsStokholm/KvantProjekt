\section{Kvantisering}
Vi antager nu at vi har det klassiske vendepunkt, hvor $E > V(x) $ for alle $ x$. Vi siger nu at vores bølgefunktion kun må eksistere på x-intervallet [a,b], hvilket medfører at vores sandsynlighedstæthed skal være 0 udenfor dette interval, altså $|\psi|^2 = 0$. Dette medfører at $\psi = 0$
Ser vi på bølgefunktionen, $\psi$, ser vi at vi kan skrive den som to dele:

\begin{equation}
  \psi(x) \cong \frac{1}{\sqrt{p(x)}}\left[C_1e^{i\phi(x)}+C_2e^{-i\phi(x)}\right]
  \label{eq:kvantiseringStart}
\end{equation}

Her ses det at vi kan skrive dette om, ved at definere to nye konstanter, $C_3 \equiv i(C_1-C_2)$ og $C_4 \equiv C_1+C_2$.

\begin{equation}
  \psi(x) = \frac{1}{\sqrt{p(x)}}
  \Bigl[    C_3\sin{\phi(x)}+C_4\cos{\phi(x)}   \Bigr]
  \label{eq:kvantiseringSinCos}
\end{equation}
Hvor det gælder at $\phi(x) = \int p(x) dx$, og da vi er i de klassiske vendepunkter når $ x = a$ eller $x = b$ så gælder det at $p(x) = 0$. Altså vil $\phi(a) = \phi(b) = 0$. Altså finder vi at
\begin{equation}
  \psi(a) = 0 = \frac{1}{\sqrt{p(x)}}\Bigl[C_3\sin(0) + C_4\cos(0)     \Bigr]
  \label{eq:kvantiseringGraense}
\end{equation}
Dette nødvendiggør at $C_4 = 0$, hvorfor $\psi$
