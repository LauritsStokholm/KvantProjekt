\section{The Hydrogen Atom}
Den radiale Schrödingerligningen er givet ved \cite{griffiths}
%
\begin{align}
    \frac{-\hbar^{2}}{2m}\diff[2]{u(r)}{r} + \left( - \frac{e^{2}}{4\pi\epsilon_{0}r} + \frac{\hbar^{2}}{2m}\frac{l(l+1)}{r^{2}} \right)u(r) & \nonumber \\
    = Eu(r), &
    \label{eq:schroedinger3d}
\end{align}

hvor $V_{\text{eff}}$ er en sum af Coulomb potentialet og et led tilhørende et angulært moment.

Fra \cite{griffiths} kendes bohrradien, $a_{0}$ og Rydbergenergien $\mathrm{Ry}$ til følgende værdier;

\begin{align}
    a_{0} = & \frac{\hbar^{2}}{2m}\left( \frac{e^{2}}{4\pi\epsilon_{0}} \right) \\
\mathrm{Ry} = & \frac{m}{2\hbar^{2}}\left( \frac{e^{2}}{4\pi\epsilon_0} \right)^{2}
\label{eq:konstanter}
\end{align}

Sættes energien til at være et multiplum af Rydbergenergien, $E = -\varepsilon \mathrm{Ry}$, og genkendes det, at $a_{0}^{2}\mathrm{Ry} = \frac{\hbar^{2}}{2m}$, samt at $2a_{0}\mathrm{Ry} = \frac{e^{2}}{4\pi\epsilon_{0}}$, kan \fxnote{reference} skrives
\begin{align}
    k(r) = & \frac{1}{\hbar} \sqrt{2m\left( E - V_{\text{eff}} \right)}\\
    = & \sqrt{\frac{1}{a_{0}^{2}\mathrm{\mathrm{Ry}}} \left( E + \frac{e^{2}}{4\pi\epsilon_{0}r} - \frac{\hbar^{2}}{2m}\frac{l(l+1)}{r^{2}} \right)}\\
    = & \sqrt{\frac{1}{a_{0}^{2}\mathrm{Ry}} \left( (-\varepsilon \mathrm{Ry}) + \frac{2a_{0}\mathrm{Ry}}{r} - a_{0}^{2}\mathrm{Ry}\frac{l(l+1)}{r^{2}} \right)}\\
    = & \frac{1}{r} \sqrt{\frac{1}{a_{0}^{2}{Ry}} \left( (-\varepsilon \mathrm{Ry})r^{2} + 2a_{0}\mathrm{Ry} r - a_{0}^{2}\mathrm{Ry} l(l+1) \right)}
\end{align}
Trækker vi her $\frac{1}{a_0}$ udenfor og ganger $\mathrm{Ry}$ ind i parantesen får vi
\begin{align}
  k(r) = & \frac{1}{a_0} \frac{1}{r} \sqrt{-\varepsilon r^2 + 2a_0r - a_0^2l(l+1)} \\
  = & \frac{\sqrt{\varepsilon}}{a_0} \frac{1}{r} \sqrt{r^2 - \frac{2a_0}{\varepsilon}r + \frac{a_0^2}{\varepsilon}l(l+1)}
\end{align}

Nu er det indre i kvadratroden på en form som kan skrives som noget $(x-a)(x-b) = x^2 -x(a+b) + ab$. Nu undersøges det om man kan ende ud med at få disse $a$ og $b$ størrelser til at være de klassiske vendepunkter, som er de $r$ som opfylder at $k(r) = 0$.
Da det udenfor kvadratroden i $k(r)$ ikke kan være 0, sættes det indre lig 0.
\begin{equation}
  r^2 - \frac{2a_0}{\varepsilon}r + \frac{a_0^2}{\varepsilon}l(l+1) = 0
\end{equation}
Dette har løsninger ved
\begin{align}
  r = & \frac{\frac{2a_0}{\varepsilon} \pm \sqrt{\frac{4a_0^2}{\varepsilon^2} - \frac{4a_0^2}{\varepsilon}l(l+1)   }  }{2}\\
    = & \frac{a_0}{\varepsilon}[1+\sqrt{1-\varepsilon l(l+1)}]
\end{align}
Herved bliver de klassiske vendepunkter
\begin{align}
  a = & \frac{a_0}{\varepsilon}\Bigl[1-\sqrt{1-\varepsilon l(l+1)}  \Bigr]\\
  b = & \frac{a_0}{\varepsilon}\Bigl[1+\sqrt{1-\varepsilon l(l+1)}  \Bigr]
\end{align}
For at det nu skal passe i vores udtryk for $k(r)$ tjekkes det om
\begin{align}
  a+b = & \frac{2a_0}{\varepsilon} \\
  ab  = & \frac{a_0^2}{\varepsilon} l(l+1)
\end{align}
$a+b$ leddet er oplagt, mens $ab$ leddet bliver
\begin{align}
  ab = & \frac{a_0^2}{\varepsilon^2}\Bigl(1-\sqrt{1-\varepsilon l(l+1)} \Bigr)\Bigl(1+\sqrt{1-\varepsilon l(l+1)} \Bigr)\\
     = & \frac{a_0^2}{\varepsilon^2}\Bigl(1-(1-\varepsilon l(l+1)) \Bigr)\\
     = & \frac{a_0^2}{\varepsilon^2} \varepsilon l(l+1)\\
     = & \frac{a_0^2}{\varepsilon} l(l+1)
\end{align}



Plottes $V_{\text{eff}}$

\begin{figure}[h!]
    \centering
    \includegraphics[width=\columnwidth]{hydrogen}
    \caption{Hydrogen}
    \label{fig:hydrogen}
\end{figure}


%
